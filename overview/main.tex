\documentclass{article}
\usepackage[english]{babel}
\usepackage[utf8]{inputenc}
\usepackage{amsmath}
\usepackage{amssymb}
\usepackage{stmaryrd}
\usepackage{changepage}
\usepackage{xcolor}
\usepackage{listings}
\usepackage{graphicx}
\usepackage{float}
\usepackage{subcaption}
\usepackage{multicol}
\usepackage{wrapfig}
%\usepackage{listings}
%\usepackage{fancyvrb}

\newcommand{\derive}[2]{\genfrac{}{}{0.5pt}{0}{#1}{#2}}
\newcommand{\assume}[2]{\genfrac{}{}{0pt}{0}{#1}{#2}}
\newcommand{\bigE}{\mathcal{E}}
\newcommand{\bigS}{\mathcal{S}}
\newcommand{\eval}{\rightarrow}
\newcommand{\reduce}[1]{\xrightarrow{#1}}
\newcommand{\reduceTwo}[2]{\xrightarrow{#1\ :\ #2}}
\newcommand{\curl}[1]{\{#1\}}
\newcommand{\bigC}{\mathbb{C}}
\newcommand{\serverRed}[4]{\langle #1 \rangle \reduceTwo{#2}{#3} \langle #4 \rangle}
\newcommand{\serverRedR}[4]{\langle #1, #2, #3 \rangle \downarrow #4}
\newcommand{\tx}[1]{\text{#1}}
\newcommand{\Rule}[3]{\derive{\begin{gathered}#1\end{gathered}}{#2}\ \textsc{#3}}
\newcommand{\setcomp}[2]{\curl{#1\ |\ #2}}
\newcommand{\unit}{\texttt{()}}
\newcommand{\typesTo}[3]{#1 \vDash #2 : #3}

\title{Safe and Secure Software-Defined Networks in P4\\
    Artifact overview}

\author{...}

\begin{document}
\maketitle

\section{Introduction}
% In the introduction, briefly explain the purpose of the artifact and how it
% supports the paper. We recommend listing all claims in the paper and stating
% whether or not each is supported. For supported claims, say how the artifact
% provides support. For unsupported claims, explain why they are omitted.
This file provides an overview of the companion artifact for the paper
\textbf{SafeP4R: a Verified API for P4 Control Plane Programs}.
The artifact consists of:

\begin{enumerate}
    \item The type-parametric API \texttt{SafeP4R} for performing
        P4Runtime operations (\texttt{connect}, \texttt{read}, \texttt{insert},
        etc.).
    \item A software tool for generating Scala 3 types from a given P4Info
        file, used for constraining the \texttt{SafeP4R} API.

    \item A virtual machine with a network simulator.
\end{enumerate}

\subsection{The \texttt{SafeP4R} API}
\texttt{SafeP4R} is a novel verified P4Runtime API for Scala 3 which statically
checks the correctness of control plane operations through types. More
specifically, the API can rule out mismatches between P4 tables, actions and
action parameters in table entries. The API supports five P4Runtime operations:
\texttt{connect}, \texttt{read}, \texttt{insert}, \texttt{modify} and
\texttt{delete}; each of these operations is layered on top of the official,
loosely-typed P4Runtime API, which is based on Protobuf RPC.

% In its current state, \texttt{SafeP4R} supports all claims of functionality
% made in the paper.

\subsection{Type generator}

The type generator takes as input a P4Info file (representing a P4 device's
tables, actions, etc.) and generates corresponding Scala 3 types which, when
used together with the SafeP4R API, guarantee type safety and progress of
P4Runtime operations.

\subsection{Virtual machine}

This artifact includes a VM with the network simulator \texttt{mininet} plus
various scripts to set up a virtual network of P4-enabled switches.  When the VM
is running, its virtual network can be queried and updated using our
\texttt{SafeP4R} API, and is used by all examples included in this artifact.

% In its current sate, the type generator supports all claims of functionality
% made in the paper.

\section{Kick-the-Tires Guide}

% In the Getting Started Guide, give instructions for setup and basic testing.
% List any software requirements and/or passwords needed to access the artifact.
% The instructions should take roughly 30 minutes to complete. Reviewers will
% follow the guide during an initial kick-the-tires phase and report issues as
% they arise.

% The Getting Started Guide should be as simple as possible, and yet it should
% stress the key elements of your artifact. Anyone who has followed the Getting
% Started Guide should have no technical difficulties with the rest of your
% artifact.

Please follow the ``Kick-the-Tires'' instructions provided in the file
\texttt{README.md} included in the artifact.

\section{Step-by-Step Instructions}

% In the Step by Step Instructions, explain how to reproduce any experiments or
% other activities that support the conclusions in your paper. Write this for
% readers who have a deep interest in your work and are studying it to improve it
% or compare against it. If your artifact runs for more than a few minutes, point
% this out, note how long it is expected to run (roughly) and explain how to run
% it on smaller inputs. Reviewers may choose to run on smaller inputs or larger
% inputs depending on available resources.
% Be sure to explain the expected outputs produced by the Step by Step
% Instructions. State where to find the outputs and how to interpret them
% relative to the paper. If there are any expected warnings or error messages,
% explain those as well. Ideally, artifacts should include sample outputs and
% logs for comparison.

To reproduce the examples in the paper and try additional experiments, please
follow the instructions provided in the file \texttt{README.md} included in the
artifact.

\bibliographystyle{plain}
\bibliography{refs}

\end{document}