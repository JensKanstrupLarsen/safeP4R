\documentclass{article}
\usepackage[english]{babel}
\usepackage[utf8]{inputenc}
\usepackage{amsmath}
\usepackage{amssymb}
\usepackage{stmaryrd}
\usepackage{changepage}
\usepackage{xcolor}
\usepackage{listings}
\usepackage{graphicx}
\usepackage{float}
\usepackage{subcaption}
\usepackage{multicol}
\usepackage{wrapfig}
%\usepackage{listings}
%\usepackage{fancyvrb}

\newcommand{\derive}[2]{\genfrac{}{}{0.5pt}{0}{#1}{#2}}
\newcommand{\assume}[2]{\genfrac{}{}{0pt}{0}{#1}{#2}}
\newcommand{\bigE}{\mathcal{E}}
\newcommand{\bigS}{\mathcal{S}}
\newcommand{\eval}{\rightarrow}
\newcommand{\reduce}[1]{\xrightarrow{#1}}
\newcommand{\reduceTwo}[2]{\xrightarrow{#1\ :\ #2}}
\newcommand{\curl}[1]{\{#1\}}
\newcommand{\bigC}{\mathbb{C}}
\newcommand{\serverRed}[4]{\langle #1 \rangle \reduceTwo{#2}{#3} \langle #4 \rangle}
\newcommand{\serverRedR}[4]{\langle #1, #2, #3 \rangle \downarrow #4}
\newcommand{\tx}[1]{\text{#1}}
\newcommand{\Rule}[3]{\derive{\begin{gathered}#1\end{gathered}}{#2}\ \textsc{#3}}
\newcommand{\setcomp}[2]{\curl{#1\ |\ #2}}
\newcommand{\unit}{\texttt{()}}
\newcommand{\typesTo}[3]{#1 \vDash #2 : #3}

\title{Safe and Secure Software-Defined Networks in P4\\
    Artifact overview}

\author{...}

\begin{document}
\maketitle

\section{Introduction}
% In the introduction, briefly explain the purpose of the artifact and how it
% supports the paper. We recommend listing all claims in the paper and stating
% whether or not each is supported. For supported claims, say how the artifact
% provides support. For unsupported claims, explain why they are omitted.
This file provides an overview of the companion artifact for the paper
\textbf{SafeP4R: a Verified API for P4 Control Plane Programs}.
The artifact consists mainly of two parts:

\begin{enumerate}
    \item The type-parametric API \texttt{SafeP4R} used for performing
        P4Runtime operations (\texttt{connect}, \texttt{read}, \texttt{insert},
        etc.).
    \item A software tool for generating Scala 3 types from a given P4Info
        file, used for constraining the \texttt{SafeP4R} API.
\end{enumerate}

\subsection{The \texttt{SafeP4R} API}
\texttt{SafeP4R} is a novel verified P4Runtime API for Scala 3 which statically
checks the correctness of control plane operations through types. More
specifically, the API can rule out mismatches between P4 tables, actions and
action parameters in table entries. The API supports five P4Runtime operations:
\texttt{connect}, \texttt{read}, \texttt{insert}, \texttt{modify} and
\texttt{delete}, each of which are layered on top of the loosely-typed P4Runtime
protobuf RPC.

In its current state, \texttt{SafeP4R} supports all claims of functionality
made in the paper.

\subsection{Type generator}
The type generator takes as input a P4Info file (representing a P4 device's
tables, actions, etc.) and generates "equivalent" Scala 3 types which can be
used together with the API to guarantee type safety.

In its current sate, the type generator supports all claims of functionality
made in the paper.

\section{Kick-the-Tires Guide}
% In the Getting Started Guide, give instructions for setup and basic testing.
% List any software requirements and/or passwords needed to access the artifact.
% The instructions should take roughly 30 minutes to complete. Reviewers will
% follow the guide during an initial kick-the-tires phase and report issues as
% they arise.

% The Getting Started Guide should be as simple as possible, and yet it should
% stress the key elements of your artifact. Anyone who has followed the Getting
% Started Guide should have no technical difficulties with the rest of your
% artifact.
In order to quickly start and test the API, use the following guide:

\begin{enumerate}
    \item Navigate to the \texttt{vm/} directory.
    \item Run \texttt{vagrant up} to build and run the VM. Building the VM will take 10-15 minutes.
    \item When the \texttt{vagrant} building procedure is complete, the VM will reboot.
       (From now on, you can launch the VM from VirtualBox, without using \texttt{vagrant} again.)
    \item When the VM presents a graphical log-in prompt:
        \begin{enumerate}
            \item Log on as user \textbf{safeP4R} with the password \texttt{safeP4R}.
            \item Open a terminal in the VM and run \texttt{make test}.
                This will start the mininet network simulation with four hosts and four switches \texttt{s1..s4}
                (see the \texttt{topology.json} file for the layout).
                It also applies the P4 configuration \texttt{config1} to \texttt{s1} and \texttt{s2},
                and \texttt{config2} to \texttt{s3} and \texttt{s4}.
        \end{enumerate}
    \item Now, navigate to the \texttt{safeP4R/} directory on the \textbf{host} machine that is running the VM.
    \item Run \texttt{sbt "runMain safeP4Rtest"}. This will run the program in\\
        \texttt{src/main/scala/examples/safeP4Rtest.scala}, which connects to the mininet network in the VM and
        sends some test queries to the \texttt{s1} switch. If everything goes well, it will print
        \texttt{Test successful!} followed by \texttt{[success]}
\end{enumerate}

\section{Step-by-Step Instructions}
% In the Step by Step Instructions, explain how to reproduce any experiments or
% other activities that support the conclusions in your paper. Write this for
% readers who have a deep interest in your work and are studying it to improve it
% or compare against it. If your artifact runs for more than a few minutes, point
% this out, note how long it is expected to run (roughly) and explain how to run
% it on smaller inputs. Reviewers may choose to run on smaller inputs or larger
% inputs depending on available resources.
% Be sure to explain the expected outputs produced by the Step by Step
% Instructions. State where to find the outputs and how to interpret them
% relative to the paper. If there are any expected warnings or error messages,
% explain those as well. Ideally, artifacts should include sample outputs and
% logs for comparison.
\textit{See the github repo?}

\bibliographystyle{plain}
\bibliography{refs}

\end{document}